\documentclass[12pt, a4paper]{article}

\usepackage{amsmath}


\def\vx{\boldsymbol{x}}
\def\vw{\boldsymbol{w}}
\def\vtheta{\boldsymbol{\theta}}


\title{Multilayer Perceptrons}
\author{CHEN Si}
\date{}

\begin{document} 

\maketitle


\section{Introduction}

    \begin{itemize}
        \item For short, \textbf{MLPs}
        \item Also called 
            \textbf{Deep Feedforward Networks} or
            \textbf{Feedforward Neural Networks}
    \end{itemize}


\section{Motivation}

    \begin{enumerate}
        \item Weaknesses of linear models
            \begin{itemize}
                \item Linearity is a strong assumption
                    \begin{itemize}
                        \item Because outputs are always monotonic with inputs
                    \end{itemize}
                \item We want to learn a representation of our data (the first $L-1$ layers), take into account the relevant interactions among our features, on top of which a linear model (the final layer) would be suitable
            \end{itemize}
        \item How to obtain the features $\phi(\vx)$ of $\vx$
            \begin{itemize}
                \item Kernel trick, such as infinite-dimensional RBF Kernel
                    \begin{itemize}
                        \item Enough capacity to fit the training set, but poor generalization
                    \end{itemize}
                \item Manually engineer $\phi$
                    \begin{itemize}
                        \item Require lots of effort for each separate task
                    \end{itemize}
            \end{itemize}
        \item From neuroscience
            \begin{itemize}
                \item The idea of using many layers of vector-valued representations is drawn from neuroscience
            \end{itemize}
        \item From mathematical and engineering disciplines
            \begin{itemize}
                \item Function approximation machines
            \end{itemize}
    \end{enumerate}


\section{Goal}

    \begin{enumerate}
        \item Deal with the weaknesses of linear models
            \begin{itemize}
                \item With deep neural networks, we used observational data to jointly learn both a representation via hidden layers and a linear predictor that acts upon that representation
                \item To capture complex interactions among our inputs via their hidden neurons
                \item To use a model that learns a different feature space in which a linear model is able to represent the solution
            \end{itemize}
        \item Learn the features $\phi(\vx;\vtheta)$
            \begin{itemize}
                \item We have a model $y = f(\vx;\vtheta, \vw) = \phi(\vx;\vtheta)^\top \vw$
            \end{itemize}
        \item \textbf{Universal Approximators}
            \begin{itemize}
                \item Designed to achieve statistical generalization (rather than to model our brain)
                \item Even with a single-hidden-layer network, given enough nodes (possibly absurdly many), and the right set of weights, we can model any function
                \item Approximate many functions much more compactly by using \textbf{deeper} (vs. \textbf{wider}) networks
                \item Learning that function is actually the hard part
            \end{itemize}
    \end{enumerate}


\section{Model}

    \subsection{Terminology}
    \begin{itemize}
        \item Layer: A function in the chain structure
        \item Hidden layers: The layers for which the training data does not show the desired output
    \end{itemize}

    \subsection{Assumptions}
    \begin{itemize}
        \item For each example, $\vx$ is accompanied by a label $y \approx f^*(\vx)$
    \end{itemize}


\section{Advantages}

    \begin{itemize}
        \item Being highly generic: broad family $\phi(\vx;\vtheta)$
        \item Generalization: only need to find the right general function family rather than finding precisely the right function
    \end{itemize}


\section{Disadvantages}

    \begin{itemize}
        \item Give up on the convexity of the training problem
    \end{itemize}


\end{document}