\documentclass[12pt, a4paper]{article}

\usepackage{amsmath}
\usepackage{amsfonts}
\usepackage{graphicx}
\usepackage{pdfpages}
\usepackage[titletoc, title]{appendix}
\usepackage{hyperref}
\hypersetup{
    colorlinks=true,
    linkcolor=blue,
    filecolor=magenta,      
    urlcolor=cyan,
}


\def\vb{\boldsymbol{b}}
\def\vh{\boldsymbol{h}}
\def\vw{\boldsymbol{w}}
\def\vx{\boldsymbol{x}}
\def\vy{\boldsymbol{y}}
\def\vz{\boldsymbol{z}}
\def\v1{\boldsymbol{1}}

\def\vI{\boldsymbol{I}}
\def\vW{\boldsymbol{W}}
\def\vX{\boldsymbol{X}}
\def\vY{\boldsymbol{Y}}

\def\vtheta{\boldsymbol{\theta}}

\def\rmx{\mathrm{x}}

\def\vrmx{\boldsymbol{\mathrm{x}}}
\def\vrmy{\boldsymbol{\mathrm{y}}}

\def\bbE{\mathbb{E}}
\def\bbX{\mathbb{X}}

\DeclareMathOperator*{\relu}{ReLU}
\DeclareMathOperator*{\argmax}{arg\,max}
\DeclareMathOperator*{\argmin}{arg\,min}
\DeclareMathOperator*{\softmax}{softmax}

\newcommand{\egvx}[1]{\boldsymbol{x}^{(#1)}}
\newcommand{\egvy}[1]{\boldsymbol{y}^{(#1)}}
\newcommand{\egx}[1]{x^{(#1)}}
\newcommand{\egy}[1]{y^{(#1)}}
\newcommand{\eghy}[1]{\hat{y}^{(#1)}}

\newcommand{\expect}[3]{\mathbb{E}_{#1 \sim #2} \left[ #3 \right]}
\newcommand{\dkl}[2]{D_{\mathrm{KL}}(#1 \Vert #2)}
\newcommand{\condiP}[3]{P(#1 \mid #2;#3)}
\newcommand{\condip}[3]{p(#1 \mid #2;#3)}
\newcommand{\ND}[3]{\mathcal{N}(#1;#2,#3)}

\newcommand{\dx}[1]{\frac{\mathrm{d}}{\mathrm{d}x} #1}


\title{Loss Functions}
\author{CHEN Si}
\date{}


\begin{document} 


\maketitle
\tableofcontents


\section{Introduction}
\begin{enumerate}
    \item In most cases, our parametric model defines a distribution $\condip{\vy}{\vx}{\vtheta}$ and we simply use the principle of maximum likelihood.
        \begin{itemize}
            \item This means we use the \textbf{cross-entropy} between the training data and the model's predictions as the cost function.
        \end{itemize}
    \item Sometimes, we merely predict some statistic of $\vy$ conditioned on $\vx$
        \begin{itemize}
            \item Specialized loss functions enable us to train a predictor of these estimates.
        \end{itemize}
    \item The total loss function used to train a neural network often combine a primary loss function with a regularization term.
\end{enumerate}


\subsection{Learning Conditional Distribution with Maximum Likelihood}
Most modern neural networks are trained using maximum likelihood.
\[
    J(\vtheta) = - \mathbb{E}_{\vrmx,\vrmy\sim\hat{\rho}_\text{data}} \log p_\text{model}(\vy \mid \vx)
\]
\begin{itemize}
    \item This means that the cost function is simply the negative log-likelihood, equivalently described as the cross-entropy between the training data and the model distribution.
\end{itemize}
\subsubsection{Advantages}
\begin{enumerate}
    \item Deriving the loss function from maximum likelihood removes the burden of designing cost functions for each model.
    \item Helps to avoid the problem of saturation
        \begin{itemize}
            \item Several output units involve an exp function that can saturate when its argument is very negative, because of the log function in the negative log-likelihood loss function.
            \item Prevent the gradient vanishing caused by saturation.
        \end{itemize}
\end{enumerate}
\subsubsection{Example}
If $p_\text{model}(\vy \mid \vx) = \ND{\vy}{f(\vx;\vtheta)}{\vI}$, then we recover the mean squared error cost
\[
    J(\vtheta) = \frac{1}{2} \mathbb{E}_{\vrmx,\vrmy\sim\hat{\rho}_\text{data}} \Vert \vy - f(\vx;\vtheta) \Vert^2 + \mathrm{const}
\]


\subsection{Learning Conditional Statistics}
\begin{itemize}
    \item Instead of learning a full probability distribution $\condip{\vy}{\vx}{\vtheta}$, we often want to learn just one conditional statistic of $\vy$ given $\vx$. (e.g. The mean of $\vy$ given $\vx$.)
    \item The loss function can be viewed as a mapping from functions (rather than a set of parameters) to real numbers.
\end{itemize}
\subsubsection{Two results derived from calculus of variations} Different cost functions give different statistics.
\begin{enumerate}
    \item Solving the optimization problem 
        \[
            f^* = \argmin_f \mathbb{E}_{\vrmx,\vrmy\sim p_\text{data}} \Vert \vy - f(\vx) \Vert^2
        \]
        yields
        \[
            f^*(\vx) = \mathbb{E}_{\vrmy\sim p_\text{data}(\vy\mid\vx)} [\vy]
        \]
        so long as this function lies within the class we optimize over.
        \begin{itemize}
            \item If we could train on infinitely many samples from the true data generating distribution, minimizing the MSE cost function would give a function that predicts the \textbf{mean} of $\vy$ for each value of $\vx$.
        \end{itemize}
    \item Solving the optimization problem
        \[
            f^* = \argmin_f \mathbb{E}_{\vrmx,\vrmy\sim p_\text{data}} \Vert \vy - f(\vx) \Vert_1
        \]
        yields a function that predicts the \textbf{median} value of $\vy$ for each $\vx$, as long as such a function may be described by the family of functions we optimize over.
        \begin{itemize}
            \item This cost function is commonly called \textbf{mean absolute error}.
        \end{itemize}
\end{enumerate}
\paragraph{Disadvantages}
\begin{itemize}
    \item MSE and MAE often lead to poor results when used with gradient-based optimization. Some output units that saturate produce very small gradients when combined with these cost functions.
    \begin{itemize}
        \item This is one reason that the cross-entropy cost function (negative log-likelihood) is more popular than MSE or MAE, even when it is not necessary to estimate an entire distribution $p(\vy \mid \vx)$.
    \end{itemize}
\end{itemize}


\section{Output Units}
The choice of how to represent the ouput then determines the form of the cross-entropy function.
\newline
Throughout this section, we suppose that the feedforward network provides a set of hidden features defined by $\vh = f(\vx;\vtheta)$.
\subsection{Linear Units}
\paragraph{Definition}
\[
    \hat{\vy} = \vW^\top\vh+\vb
\]
\begin{itemize}
    \item Linear output layers are often used to produce the mean of a conditional Gaussian distribution
        \[
            p(\vy \mid \vx) = \ND{\vy}{\hat{\vy}}{\vI}
        \]
\end{itemize}
\paragraph{Advantages}
\begin{enumerate}
    \item Because linear units do not saturate, they pose little difficulty for gradient-based optimization algorithms.
\end{enumerate}
\subsection{Sigmoid Units for Bernoulli Output Distributions}
\paragraph{Definition}
\[
    \hat{y} = \sigma \left( \vw^\top \vh + b \right)
\]
\begin{itemize}
    \item $z = \vw^\top \vh + b$ defining such a distribution over binary variables is called a \textbf{logit}.
    \item Predicts only $P(y=1 \mid \vx)$ 
        \newline
        (Actually predicts unnormalized log probability $\log \tilde{P}(y=1 \mid \vx) = z$)
\end{itemize}
\paragraph{Motivation}
Construct an unnormalized probability distribution $\tilde{P}(y)$. $\forall y=0,1$, if we assume 
\[
    \log \tilde{P}(y) = yz
\]
After exponentiating we obtain
\[
    \tilde{P}(y) = \exp (yz)
\]
After normalizing we obtain 
\[
    \begin{split}
        P(y) &= \frac{\exp(yz)}{\sum_{y'=0}^1 \exp(y'z)}
        \\&= \sigma ((2y-1)z)
    \end{split}
\]
\paragraph{Advantages}
\begin{enumerate}
    \item \textbf{Ensures there is always a strong gradient whenever the model has the wrong answer.}
    \item Prevent the saturation of Sigmoid if we use MLE.
        \[
            \begin{split}
                J(\vtheta) &= - \log P(y \mid \vx)
                \\&= - \log \sigma((2y-1)z)
                \\&= \zeta((1-2y)z)
            \end{split}
        \]
        Thus, gradient-based learning can act to quickly correct a mistaken $z$, because the softplus function asymptotes toward simply returning its argument $(1-2y)z = \vert z \vert$ if $z$ have the wrong sign.
    \item MLE is almost always the preferred approach to training sigmoid output units.
    \begin{itemize}
        \item If we use MSE, the gradient can shrink too small whether the model has the correct answer or thr incorrect answer.
    \end{itemize}
\end{enumerate}
\paragraph{In Software Implementation}
\begin{enumerate}
    \item Write the negative log-likelihood as a function of $z$, rather than as a function of $\hat{y} = \sigma(z)$
    \begin{itemize}
        \item To avoid numerical underflow of Sigmoid, which yields numerical overflow of logarithm.
    \end{itemize}
\end{enumerate}
\paragraph{Applications}
\begin{enumerate}
    \item Binary classifier.
\end{enumerate}


\subsection{Softmax Units for Multinoulli Output Distributions}
\paragraph{Definition}
First, predicts unnormalized log probabilities
\[
    \vz = \vW^\top\vh + \vb
\]
where
\[
    z_i = \log \tilde{P}(y=i \mid \vx)
\]
The softmax function can then exponentiate and normalize $\vz$ to obtain the desired $\hat{\vy}$
\[
    \softmax(\vz)_i = \frac{\exp(z_i)}{\sum_j \exp(z_j)}
\]
\begin{itemize}
    \item $\forall i \in 1, 2, \dots, n,\ \softmax(\vz)_i \in (0, 1)$
    \item $\displaystyle \sum_i \softmax(\vz)_i = 1$
\end{itemize}
\paragraph{Properties}
\begin{enumerate}
    \item Invariant to adding the same scalar to all its inputs
        \[
            \softmax(\vz) = \softmax(\vz + c)
        \]
    \item A "softened" (continuous and differentiable) version of the $\argmax$ (with its result represented as a one-hot vector).
    \item The corresponding soft version of the maximum function is $\softmax(\vz)^\top \vz$.
\end{enumerate}
\paragraph{Advantages}
\begin{enumerate}
    \item As with the logistic sigmoid, the use of the exp function works well when training the softmax to output a target value $y$ using maximum log-likelihood.
    \item The log in the log-likelihood undoes the exp of the softmax
        \[
            \log \softmax(\vz)_i = z_i - \log \sum_j \exp(z_j)
        \]
        \begin{itemize}
            \item The fist term shows that the input $z_i$ always has a direct contribution to the cost funtion because this term cannot saturate.
            \item $\log \sum_j \exp(z_j) \approx \max_j z_j$
                \begin{itemize}
                    \item Always penalizes the most active incorrect prediction.
                    \item If the correct answer already has the largest input to the softmax, then the two terms will roughly cancel.
                \end{itemize}
            \item When maximizing the log-likelihood, the first term encourages $z_i$ to be pushed up, while the second term encourages all of $\vz$ to be pushed down.
        \end{itemize}
    \item Unregularized maximum likelihood will drive the model to learn parameters that drive the softmax to predict the fraction of counts of each outcome observed in the training set
        \[
            \softmax(\vz(\vx;\vtheta))_i \approx \frac{\sum_{j=1}^m \v1_{\egy{j}=i, \egvx{j}=\vx}}{\sum_{j=1}^m \v1_{\egvx{j}=\vx}}
        \]
        \begin{itemize}
            \item Because MLE is a consistent extimator, this if guaranteed to happen as long as the model family is capable of representing the training distribution.
        \end{itemize}
    \item Squared error (or other loss function with out $\log$) is a poor loss function for softmax units and can fail to train the model to change its output, even when the model makes highly confident incorrect predictions.
        \begin{itemize}
            \item The softmax activation can saturate when the differences between input values become extreme.
                \begin{itemize}
                    \item An output $\softmax(\vz)_i$ saturates to $1$ when the corresponding input is maximal ($z_i = \max_i z_i$) and $z_i$ is much greater than all the other inputs.
                    \item An output $\softmax(\vz)_i$ saturates to $0$ when the corresponding input $z_i$ is not maximal and the maximum is much greater.
                \end{itemize}
            \item The gradient will vanish when the argument to the exp becomes very negative.
        \end{itemize}
\end{enumerate}
\paragraph{Applications}
\begin{enumerate}
    \item Used as the output of a multiclass-classifier, to represent the probability distribution over $n$ different classes.
    \item More rarely, used inside the model itself, if we wish the model to choose between one of $n$ different options for some internal variable.
\end{enumerate}


\subsection{Other Output Types}
(To be accomplished) (See \textit{Deep Learning Book 6.2.2.4})


\end{document}